 %&latex
\documentclass[smaller,a4paper]{beamer}
\usepackage{amsmath,amssymb,pdfsync,listings}
\usepackage{graphicx}
\usepackage{truncate}
%%\usepackage{mpmulti}
\usepackage{times}

\newcommand{\largenameref}[1]{\textbf{\huge{\nameref{#1}}}}
\newcommand{\largehyperlink}[2]{\textbf{\Large\hyperlink{#1}{#2} }}
\newcommand{\largehypertarget}[2]{\textbf{\color{red} \hypertarget{#1}{#2} }}
\newcommand{\Largehypertarget}[2]{\textbf{\Large \color{red} \hypertarget{#1}{#2} }}
\newcommand{\Int}[2]{\displaystyle{\int\limits_{#1}^{#2}}}      
\newcommand{\Sum}[2]{\displaystyle{\sum\limits_{#1}^{#2}}}      
\newcommand{\Myfoilheadskip}[1]{\begin{frame}\frametitle{#1}}
\newcommand{\trace}[2]{\left. #1 \right\rvert_{_{#2} }  }
\newcommand{\mtrx}[1]{\mathbf{#1}}
\newcommand{\vect}[1]{\mathbf{#1}}
\newcommand{\abs}[1]{\left|#1\right|}

\usepackage[english]{babel}
\lstset{
  language=[ISO]C++,                       % The default language
  basicstyle=\scriptsize,                  % The basic style
  extendedchars=true                       % Allow extended characters
}
\newcommand{\cpp}[1]{\lstinline!#1!}

\begin{document}
\title{Refactoring and Reorganization of the FEM1D Code}
\frame{\titlepage}


\begin{frame}[fragile]\frametitle{Exercise}
\begin{itemize}\small
\item refactor the code by creating a separate class template describing the FEM problem
\item template parameter defining the type of real numbers:
\begin{lstlisting}
template<class prec = double>
fem_1d { ... }
\end{lstlisting}
\item constructor with one argument defining the mesh, store the mesh as a \cpp{unique_ptr}
\begin{lstlisting}
fem_1d problem (std::unique_ptr<mesh>(new mesh (...)));
\end{lstlisting}
\item methods to set coefficients
\begin{lstlisting}
auto diff   = [](auto x) { return x * x; };
auto source = [](auto x) { return x * x * x; };
problem.set_diffusion_coefficient (diff);
problem.set_source_coefficient (source);
\end{lstlisting}
\item \cpp{fem_1d::assemble ()} method
\item \cpp{fem_1d::solve ()} method
\item methods to set BCs
\begin{lstlisting}
problem.set_dirichlet (fem_1d::left_boundary, 1);
problem.set_dirichlet (fem_1d::right_boundary, 0);
\end{lstlisting}
\item methods to access assembled arrays
\begin{lstlisting}
auto & A = problem.matrix ();
auto & b = problem.rhs ();
auto & c = problem.result ();
\end{lstlisting}
\end{itemize}
\end{frame}
\end{document}